%! Author = pfranco
%! Date = 7/3/2022

% Packages
\usepackage{amsmath}
\usepackage[pagebackref=true]{hyperref}

% Preamble
\documentclass[12pt,jou]{apa7}
\title{Analisis de sentimiento de textos financieros}
\shorttitle{Analisis de sentimiento de textos financieros}
\professor{Hernan}
\course{PLN}
\author{Pablo Franco}
\affiliation{Universidad Tecnologica Ncional}
\date{\today}
\keywords{PLN, Criptomonedas, Fianciero}
\leftheader{}

% Document
\begin{document}
\maketitle
\section{Introduccion}\label{sec:introduccion}
En el marco de la cursada de la materia Procesamiento del
Lenguaje Natural de la Universidad Tecnologica Nacional,
se realiza esta actividad para fijar conocimientos
mediante la aplicacion practica de tecnicas de analisis de
sentimiento de textos y tecnicas de recuperacion de informacion.
\subsection{Objetivo}\label{subsec:objetivo}
Clasificar un texto para identificar si el sentimiento
que un lector interpreta es positivo, negativo o neutro
dentro de un contexto financiero.
Esta informacion tiene como finalidad ayudar a
analizar el comportamiento de activos financieros.
Se utilizo como muestra activos financieros del mundo de las
criptomonedas.
\subsection{Tecnologias}\label{subsec:tecnologias}
Se utiliza para la realizacion del proyecto una implementacion
del lenguaje de programacion Python3\cite{cite_01} por ser esta tecnologia
la mas desarrollada y extendida en los campos academicos de la ciencia de datos.
La implementacion utilizada en esta practica se denomina CPython version 3.10.
Esta implementacion se puede utilizar y descargar libremente de internet.
Python3 es compatible con gran variedad de arquitecturas y sistemas operativos,
en este ensayo mas especificamente se utilizaron arquitecturas amd64/x86 sobre Windows 10 y Debian.
Durante la ejecucion del programa se observa que la huella de memoria principal necesaria
para procesar las noticias atraves del modelo predictivo se aproxima a los 8gb.
Vale mencionar la utilizacion de los modulos de Python fundamentales para la
realizacion de esta practica, HappyTransformer\cite{cite_02} (modulo de python para aplicar modelos
predictivos preentrenados)
Newspapaper\cite{cite_03} (modulo de python para extraer y normalizar el texto de una pagina web) y
NewsApi\cite{cite_04} (modulo de python que permite consultar articulos actualizados en cadenas
de noticias de todo el mundo).
Tambien se utilizo un modelo predictivo de tipo BERT llamado finbert\cite{cite_05}
el cual esta preentrenado con un set de datos de valoraciones de sentimiento
sobre noticias del mundo financiero.
\subsection{Metodo}\label{subsec:metodo}
Se escogen diferentes cadenas de noticias para utilizar como fuente
para el analisis de sentimiento de textos.
Ademas de las fuentes se escogen nombres de criptomonedas de las cuales
recuperar las noticias.
Esta informacion se utiliza para obtener los noticias por medio
del concentrador de noticias internacionales\cite{cite_06} tomando todas las noticias
relacionadas a partir del dia anterior hasta el momento actual.
Estas noticias se normalizan por medio del modulo newspaper\cite{cite_07} el cual utiliza
modelos de pln para corregir la puntuacion y los simbolos o definiciones de html
que no corresponde encontrarlos dentro del texto.
Al texto normalizado se le aplica un modelo predictivo tipo BERT\cite{cite_08}
denominado finbert para clasificarlos en positivo, negativo, neutro.
En base a estos resultados se hace un conteo de las probabilidades para
obtener un estimado del sentimiento aproximado de cada criptomoneda.
%\hyperlink{label1}{Click me!}.
%\hypertarget{label1}{I'm target}
\pagebreak
\begin{thebibliography}{99}
\bibitem{cite_01} \emph{Python3} - \url{https://www.python.org/}
\bibitem{cite_02} \emph{Pypi module happytransformer} - \url{https://pypi.org/project/newspaper3k/}
\bibitem{cite_03} \emph{Pypi module newspaper3k} - \url{https://pypi.org/project/newspaper3k/}
\bibitem{cite_04} \emph{Pypi module newsapi-python} - \url{https://pypi.org/project/newsapi-python/}
\bibitem{cite_05} \emph{ProsusAi/finbert} - \url{https://huggingface.co/ProsusAI/finbert}
\bibitem{cite_06} \emph{NewsApi usage} - \url{https://newsapi.org/docs/client-libraries/python}
\bibitem{cite_07} \emph{Newspaper usage} - \url{https://newspaper.readthedocs.io/en/latest/}
\bibitem{cite_08} \emph{HappyTransformer usage} - \url{https://happytransformer.com/text-classification/usage/}
\end{thebibliography}
\end{document}